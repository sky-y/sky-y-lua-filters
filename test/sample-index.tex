% Options for packages loaded elsewhere
\PassOptionsToPackage{unicode}{hyperref}
\PassOptionsToPackage{hyphens}{url}
%
\documentclass[
  a4paper,
  lualatex,
  ja=standard]{bxjsarticle}
\usepackage{lmodern}
\usepackage{amsmath}
\usepackage{ifxetex,ifluatex}
\ifnum 0\ifxetex 1\fi\ifluatex 1\fi=0 % if pdftex
  \usepackage[T1]{fontenc}
  \usepackage[utf8]{inputenc}
  \usepackage{textcomp} % provide euro and other symbols
  \usepackage{amssymb}
\else % if luatex or xetex
  \usepackage{unicode-math}
  \defaultfontfeatures{Scale=MatchLowercase}
  \defaultfontfeatures[\rmfamily]{Ligatures=TeX,Scale=1}
\fi
% Use upquote if available, for straight quotes in verbatim environments
\IfFileExists{upquote.sty}{\usepackage{upquote}}{}
\IfFileExists{microtype.sty}{% use microtype if available
  \usepackage[]{microtype}
  \UseMicrotypeSet[protrusion]{basicmath} % disable protrusion for tt fonts
}{}
\makeatletter
\@ifundefined{KOMAClassName}{% if non-KOMA class
  \IfFileExists{parskip.sty}{%
    \usepackage{parskip}
  }{% else
    \setlength{\parindent}{0pt}
    \setlength{\parskip}{6pt plus 2pt minus 1pt}}
}{% if KOMA class
  \KOMAoptions{parskip=half}}
\makeatother
\usepackage{xcolor}
\IfFileExists{xurl.sty}{\usepackage{xurl}}{} % add URL line breaks if available
\IfFileExists{bookmark.sty}{\usepackage{bookmark}}{\usepackage{hyperref}}
\hypersetup{
  hidelinks,
  pdfcreator={LaTeX via pandoc}}
\urlstyle{same} % disable monospaced font for URLs
\setlength{\emergencystretch}{3em} % prevent overfull lines
\providecommand{\tightlist}{%
  \setlength{\itemsep}{0pt}\setlength{\parskip}{0pt}}
\setcounter{secnumdepth}{-\maxdimen} % remove section numbering
% ruby
\usepackage{pxrubrica}

% tcolorbox
\usepackage{tcolorbox}
\tcbuselibrary{xparse}
\tcbuselibrary{skins}
\tcbuselibrary{breakable}

% enjoy :)
\usepackage{scsnowman}
% index
\usepackage{makeidx}
\makeindex
\ifluatex
  \usepackage{selnolig}  % disable illegal ligatures
\fi

\author{}
\date{}

\begin{document}

\hypertarget{ux6e96ux5099}{%
\subsection{準備}\label{ux6e96ux5099}}

テンプレート変数として、次のヘッダ(プリアンブル)を読み込む。

\begin{verbatim}
\usepackage{makeidx}
\makeindex
\end{verbatim}

\begin{itemize}
\tightlist
\item
  パターン1:ヘッダファイルとして読み込む

  \begin{itemize}
  \tightlist
  \item
    \texttt{pandoc\ ...\ -H\ header.tex}
  \end{itemize}
\item
  パターン2:コマンドラインで直接指定する

  \begin{itemize}
  \tightlist
  \item
    \texttt{pandoc\ ...\ -V\ header-includes="\textbackslash{}usepackage\{makeidx\}"\ -V\ header-includes="\textbackslash{}makeindex"}
  \end{itemize}
\item
  パターン3:デフォルトファイルに指定する

  \begin{itemize}
  \tightlist
  \item
    \texttt{pandoc\ ...\ -d\ defaults.yaml}
  \end{itemize}
\end{itemize}

\begin{verbatim}
variables:
  header-includes: |
    \usepackage{makeidx}
    \makeindex
\end{verbatim}

\hypertarget{ux6ce8ux610f}{%
\subsection{注意}\label{ux6ce8ux610f}}

テンプレート変数の指定が重複する場合は要注意。

\begin{itemize}
\tightlist
\item
  テンプレート変数同士は、重ねると追記される(上書きしない)

  \begin{itemize}
  \tightlist
  \item
    ヘッダファイル読み込み(\texttt{-H}/\texttt{-\/-include-in-header})でも、変数の直接指定(\texttt{-V\ header-includes=})でも同じ
  \item
    同じヘッダ内容を複数回読まないように注意
  \end{itemize}
\item
  メタデータとテンプレートの組み合わせでは、テンプレートがメタデータを上書きする
\end{itemize}

\hypertarget{ux4f8b}{%
\subsection{例}\label{ux4f8b}}

1 {Pandoc\index{Pandoc}}

2 {Pandoc's Markdown\index{Pandoc's Markdown}}

3 {けいおん!\index{けいおん"!}}

4 {aaa@bbb\index{aaa"@bbb}}

5 {ccc\textbar ddd\index{ccc\textbar{}ddd}}

6 {免疫\index{めんえき@免疫}}

7(only) {\index{LaTeX}}

8(only) {\index{へんすう@変数}}

9 {mendex\index{mendex}}

10(raw) mendexの使い方{\index{mendex!のつかいかた@---の使い方}}

11(keyword) {\textbf{Markdown}\index{Markdown}}

12(keyword) {\textbf{技術書典}\index{ぎじゅつしょてん@技術書典}}

13 {-\/-standalone\index{--standalone}}

14(code) {\texttt{-\/-template}\index{--template@\texttt{--template}}}

15(code only) {\index{--lua-filter@\texttt{--lua-filter}}}

16(raw\_tex) -\/-filter\index{--filter@\texttt{--filter}}

17(only)
\scsnowman[snow,hat=true,muffler=red,arms=true,scale=3]{\index{本質@}}

\phantomsection

\addcontentsline{toc}{chapter}{\indexname}

\printindex

\end{document}
