% Options for packages loaded elsewhere
\PassOptionsToPackage{unicode}{hyperref}
\PassOptionsToPackage{hyphens}{url}
%
\documentclass[
  a4paper,
  lualatex,
  ja=standard]{bxjsarticle}
\usepackage{lmodern}
\usepackage{amsmath}
\usepackage{ifxetex,ifluatex}
\ifnum 0\ifxetex 1\fi\ifluatex 1\fi=0 % if pdftex
  \usepackage[T1]{fontenc}
  \usepackage[utf8]{inputenc}
  \usepackage{textcomp} % provide euro and other symbols
  \usepackage{amssymb}
\else % if luatex or xetex
  \usepackage{unicode-math}
  \defaultfontfeatures{Scale=MatchLowercase}
  \defaultfontfeatures[\rmfamily]{Ligatures=TeX,Scale=1}
\fi
% Use upquote if available, for straight quotes in verbatim environments
\IfFileExists{upquote.sty}{\usepackage{upquote}}{}
\IfFileExists{microtype.sty}{% use microtype if available
  \usepackage[]{microtype}
  \UseMicrotypeSet[protrusion]{basicmath} % disable protrusion for tt fonts
}{}
\makeatletter
\@ifundefined{KOMAClassName}{% if non-KOMA class
  \IfFileExists{parskip.sty}{%
    \usepackage{parskip}
  }{% else
    \setlength{\parindent}{0pt}
    \setlength{\parskip}{6pt plus 2pt minus 1pt}}
}{% if KOMA class
  \KOMAoptions{parskip=half}}
\makeatother
\usepackage{xcolor}
\IfFileExists{xurl.sty}{\usepackage{xurl}}{} % add URL line breaks if available
\IfFileExists{bookmark.sty}{\usepackage{bookmark}}{\usepackage{hyperref}}
\hypersetup{
  hidelinks,
  pdfcreator={LaTeX via pandoc}}
\urlstyle{same} % disable monospaced font for URLs
\setlength{\emergencystretch}{3em} % prevent overfull lines
\providecommand{\tightlist}{%
  \setlength{\itemsep}{0pt}\setlength{\parskip}{0pt}}
\setcounter{secnumdepth}{-\maxdimen} % remove section numbering
% ruby
\usepackage{pxrubrica}

% tcolorbox
\usepackage{tcolorbox}
\tcbuselibrary{xparse}
\tcbuselibrary{skins}
\tcbuselibrary{breakable}
\ifluatex
  \usepackage{selnolig}  % disable illegal ligatures
\fi

\author{}
\date{}

\begin{document}

\hypertarget{latexux306eux30ebux30d3ux7528ux30d5ux30a3ux30ebux30bf}{%
\section{LaTeXのルビ用フィルタ}\label{latexux306eux30ebux30d3ux7528ux30d5ux30a3ux30ebux30bf}}

下記のLaTeXパッケージに依存する。

\begin{itemize}
\tightlist
\item
  pxrubrica

  \begin{itemize}
  \tightlist
  \item
    \texttt{\textbackslash{}usepackage\{pxrubrica\}}
    をヘッダ(プリアンブル)に読み込む
  \item
    詳細

    \begin{itemize}
    \tightlist
    \item
      \href{https://qiita.com/zr_tex8r/items/42466cbcbeb670a3a2dc}{LaTeX
      文書で``美しい日本の''ルビを使う ~pxrubrica パッケージ~ - Qiita}
    \item
      マニュアル
      \href{http://texdoc.net/texmf-dist/doc/platex/pxrubrica/pxrubrica.pdf}{pxrubrica
      パッケージ}
    \end{itemize}
  \end{itemize}
\end{itemize}

※ luatexja-ruby パッケージには非対応。

\hypertarget{ux69cbux6587}{%
\subsection{構文}\label{ux69cbux6587}}

\begin{verbatim}
[親文字](ルビ文字){.ruby}
[alphabet](欧文ルビ文字){.aruby}  ※ pxrubricaのみ
\end{verbatim}

オプションはPandoc's Markdownの属性(attribute)として与える。
オプションの構文自体は両パッケージとも共通に使える。

構文(Pandoc's Markdown):

\begin{verbatim}
[親文字](ルビ){.ruby opt="オプション"}
\end{verbatim}

たとえば

\begin{verbatim}
[雲雀](ひばり){.ruby opt="g"}
\end{verbatim}

は

\begin{verbatim}
\ruby[g]{雲雀}{ひばり}
\end{verbatim}

に変換される。

\hypertarget{ux4f8b}{%
\subsection{例}\label{ux4f8b}}

\hypertarget{ux4f8b1}{%
\subsubsection{例1}\label{ux4f8b1}}

Markdown:

\begin{verbatim}
あれは[鷹](たか){.ruby}ではなく[鶯](うぐいす){.ruby}です。
\end{verbatim}

出力結果:

あれは{\ruby{鷹}{たか}}ではなく{\ruby{鶯}{うぐいす}}です。

\hypertarget{ux4f8b2}{%
\subsubsection{例2}\label{ux4f8b2}}

Markdown:

\begin{verbatim}
[小鳩](こ|ばと){.ruby} [孔雀](く|じゃく){.ruby} [七面鳥](しち|めん|ちょう)
\end{verbatim}

出力結果:

{\ruby{小鳩}{こ|ばと}} {\ruby{孔雀}{く|じゃく}}
{\ruby{七面鳥}{しち|めん|ちょう}}

\hypertarget{ux4f8b3ux570fux70b9}{%
\subsubsection{例3(圏点)}\label{ux4f8b3ux570fux70b9}}

Markdown:

\begin{verbatim}
[本質]{.kenten}
\end{verbatim}

出力結果:

{\kenten{本質}}

\hypertarget{ux4f8b4ux30b0ux30ebux30fcux30d7ux30ebux30d3}{%
\subsubsection{例4(グループルビ)}\label{ux4f8b4ux30b0ux30ebux30fcux30d7ux30ebux30d3}}

Markdown:

\begin{verbatim}
[雲雀](ひ|ばり){.ruby opt="g"} [不如帰](ほととぎす){.ruby opt="g"}
\end{verbatim}

出力結果:

{\ruby[g]{雲雀}{ひばり}} {\ruby[g]{不如帰}{ほととぎす}}

\hypertarget{ux4f8b5ux30e2ux30ceux30ebux30d3}{%
\subsubsection{例5(モノルビ)}\label{ux4f8b5ux30e2ux30ceux30ebux30d3}}

Markdown:

\begin{verbatim}
[孔雀](く|じゃく){.ruby opt="m"} [七面鳥}(しち|めん|ちょう){.ruby opt="m"}
\end{verbatim}

出力結果:

{\ruby[m]{孔雀}{く|じゃく}}
{[}七面鳥\}(しち\textbar めん\textbar ちょう)\{.ruby opt="m"\}

\hypertarget{ux4f8b6arubyux6b27ux6587ux7528ux306eux30ebux30d3}{%
\subsection{\texorpdfstring{例6(\texttt{\textbackslash{}aruby}:欧文用のルビ)}{例6(\textbackslash aruby:欧文用のルビ)}}\label{ux4f8b6arubyux6b27ux6587ux7528ux306eux30ebux30d3}}

pxrubricaにおける\texttt{\textbackslash{}aruby}(欧文用のルビ)の構文も用意している。

構文:

\begin{verbatim}
[alphabet](ルビ){.aruby}
\end{verbatim}

Markdown:

\begin{verbatim}
[Pandoc](パンドック){.aruby}と[Markdown](マークダウン){.aruby}
\end{verbatim}

出力結果:

{\aruby{Pandoc}{パンドック}}と{\aruby{Markdown}{マークダウン}}

注意:\texttt{\textbackslash{}truby}や\texttt{\textbackslash{}atruby}などは提供しない。

\end{document}
