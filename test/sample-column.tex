% Options for packages loaded elsewhere
\PassOptionsToPackage{unicode}{hyperref}
\PassOptionsToPackage{hyphens}{url}
%
\documentclass[
  a4paper,
  lualatex,
  ja=standard]{bxjsarticle}
\usepackage{lmodern}
\usepackage{amsmath}
\usepackage{ifxetex,ifluatex}
\ifnum 0\ifxetex 1\fi\ifluatex 1\fi=0 % if pdftex
  \usepackage[T1]{fontenc}
  \usepackage[utf8]{inputenc}
  \usepackage{textcomp} % provide euro and other symbols
  \usepackage{amssymb}
\else % if luatex or xetex
  \usepackage{unicode-math}
  \defaultfontfeatures{Scale=MatchLowercase}
  \defaultfontfeatures[\rmfamily]{Ligatures=TeX,Scale=1}
\fi
% Use upquote if available, for straight quotes in verbatim environments
\IfFileExists{upquote.sty}{\usepackage{upquote}}{}
\IfFileExists{microtype.sty}{% use microtype if available
  \usepackage[]{microtype}
  \UseMicrotypeSet[protrusion]{basicmath} % disable protrusion for tt fonts
}{}
\usepackage{xcolor}
\IfFileExists{xurl.sty}{\usepackage{xurl}}{} % add URL line breaks if available
\IfFileExists{bookmark.sty}{\usepackage{bookmark}}{\usepackage{hyperref}}
\hypersetup{
  hidelinks,
  pdfcreator={LaTeX via pandoc}}
\urlstyle{same} % disable monospaced font for URLs
\setlength{\emergencystretch}{3em} % prevent overfull lines
\providecommand{\tightlist}{%
  \setlength{\itemsep}{0pt}\setlength{\parskip}{0pt}}
\setcounter{secnumdepth}{-\maxdimen} % remove section numbering
% ruby
\usepackage{luatexja-ruby}

% tcolorbox
\usepackage{tcolorbox}
\tcbuselibrary{xparse}
\tcbuselibrary{skins}
\tcbuselibrary{breakable}
\ifluatex
  \usepackage{selnolig}  % disable illegal ligatures
\fi

\author{}
\date{}

\begin{document}

\tcbset{
  breakable,
  enhanced,
  fonttitle=\bfseries,
  parbox=false,
  attach boxed title to top left={xshift=3mm, yshift*=-\tcboxedtitleheight/2},
  colbacktitle=gray!97,
}

\hypertarget{luaux30d5ux30a3ux30ebux30bf-tcolorbox-column.lua-ux306eux4f8b}{%
\section{Luaフィルタ tcolorbox-column.lua
の例}\label{luaux30d5ux30a3ux30ebux30bf-tcolorbox-column.lua-ux306eux4f8b}}

\begin{verbatim}
::: box
タイトルのない囲みです。
:::
\end{verbatim}

\begin{tcolorbox}

タイトルのない囲みです。

\end{tcolorbox}

\begin{verbatim}
::: {.box title=タイトル付きの囲み}
タイトル付きの囲みです。
:::
\end{verbatim}

\begin{tcolorbox}[title=タイトル付きの囲み,]

タイトル付きの囲みです。

\end{tcolorbox}

\begin{verbatim}
::: {.column title="何らかの小話"}
タイトルの頭に `■コラム: ` (または`column-prefix`メタデータで指定された文字列)が付きます。
:::
\end{verbatim}

\begin{tcolorbox}[title=■コラム:~何らかの小話,]

タイトルの頭に \texttt{■コラム:}
(または\texttt{column-prefix}メタデータで指定された文字列)が付きます。

\end{tcolorbox}

\begin{verbatim}
::: {.column title="何らかの小話" option="colbacktitle=black"}
オプション付きです。
:::
\end{verbatim}

\begin{tcolorbox}[title=■コラム:~何らかの小話,colbacktitle=black]

オプション付きです。

\end{tcolorbox}

\end{document}
